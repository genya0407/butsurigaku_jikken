\documentclass[uplatex,11pt]{jsarticle}

\usepackage[dvipdfmx]{graphicx}
\usepackage{comment}
%\usepackage{listings, jlisting}
\usepackage{amsmath, amssymb}
\usepackage{wrapfig}
\usepackage{fancybox}
\usepackage{ascmac}
\usepackage{subfig}
\usepackage{rotating}

\setlength\floatsep{5pt}
\setlength\textfloatsep{5pt}
\setlength\intextsep{5pt}
\setlength\abovecaptionskip{0pt}

\newcommand{\fg}[3]{ % \fg{label}{path}{caption},
    \begin{figure}
        \begin{center}
            \includegraphics[width=\textwidth]{#2}
            \caption{#3}
            \label{fg_#1}
        \end{center}
    \end{figure}
}

\newcommand{\dummyfig}[2]{
    \begin{figure}
        \begin{center}
            \begin{shadebox}
                巻末に添付
            \end{shadebox}
            \caption{#2}
            \label{fg_#1}
        \end{center}
    \end{figure}
}

\newcommand{\cir}[3]{ % \cir{label}{path}{caption},
    \begin{figure}
        \begin{center}
            \includegraphics[width=0.85\textwidth]{#2}
            \caption{#3}
            \label{fg_#1}
         \end{center}
    \end{figure}
}

\newcommand{\subtabs}[2]{
    \input{|"ruby ./convert_csv.rb #1 1 '#2'"}
}

\newcommand{\tab}[4]{ % \tab{label}{csv-path}{caption}{分割数}
    \begin{table}[htb]
        \centering
        \caption{#3}
        \input{|"ruby ./convert_csv.rb #2 #4"}
        \label{tab_#1}
    \end{table}
}

\newcommand{\fr}[1]{図\ref{fg_#1}}
\newcommand{\tr}[1]{表\ref{tab_#1}}
\newcommand{\er}[1]{式(\ref{#1})}

\begin{document}

\section{目的}

プリズム分光器を用いて水素原子スペクトルの波長を測定する。

\section{データ}

Na, Hg, Cdの既知のスペクトルに関して、波長とスケールの関係を調べて表にしたのが\tr{data}である。
これをグラフにプロットして分散曲線を描いたのが\fr{bunsan}である。

\tab{data}{csv/bunsan.csv}{実験データ}{2}

また、4本の水素スペクトルを観測してそのスケールを調べたところ以下のようであった。

\begin{itemize}
  \item 1.87
  \item 3.20
  \item 4.05
  \item 4.59
\end{itemize}

\section{解析}

\tr{data}をグラフにプロットして分散曲線を描いたのが\fr{bunsan}である。図から、分散曲線の幅は高々$\Delta\lambda = 5[{\rm nm}]$程度と読み取れる。

また、この分散曲線を用いて4本の水素スペクトルの波長を読み取ると以下のようになる。

\begin{itemize}
    \item $662 \pm 5 [{\rm nm}]$
    \item $486 \pm 5 [{\rm nm}]$
    \item $432 \pm 5 [{\rm nm}]$
    \item $410 \pm 5 [{\rm nm}]$
\end{itemize}

また、この4本のスペクトルの波長に対して上から$n=3,4,5,6$を割り当て、横軸に$\frac{1}{n^2}$を縦軸に$\frac{1}{\lambda}$を取ってグラフにプロットし、回帰曲線を描いたのが\fr{n-lambda}である。

回帰曲線の傾きは$-1.10 \times 10^7$が最ももっともらしく、最大で$-1.08\times 10^7$、最小で$-1.11\times 10^7$である。
また、回帰曲線の切片は$2.75\times 10^6$が最ももっとらしく、最大で$2.78\times 10^6$、最小で$2.71\times 10^6$である。

また、\tr{data}から縦軸に$\frac{1}{\lambda^2}$を、横軸にスケールをプロットしたのが\fr{lambda-inverse}である。
点が概ね直線上にあることがわかる。

\dummyfig{bunsan}{分散曲線}
\dummyfig{n-lambda}{$\frac{1}{\lambda}$と$\frac{1}{n^2}$の関係}
\fg{lambda-inverse}{graph/lambda_square.png}{$\frac{1}{\lambda^2}$とスケールの関係}

\section{考察}

Rydberg定数を$R_H$とすると、回帰曲線の傾きは$-R_H$に等しい。従って、回帰曲線の傾きから$R_H$を求めると、

\[
    R_H = (1.10 \pm 0.02) \times 10^7 [{\rm m^{-1}}]
\]

と求まる。教科書によると参照値は$R_H = 1.09678 \times 10^7$であり、これは誤差の範囲で一致している。

また、回帰曲線の切片は$R_H/4$に等しい。このことから、

\begin{eqnarray*}
    R_H & = & 4 \times (2.75 \pm 0.04) \times 10^6 \\
        & = & (1.1 \pm 0.2) \times 10^7 [{\rm m^{-1}}]
\end{eqnarray*}

と求まる。教科書によると参照値は$R_H = 1.09678 \times 10^7$であり、これは誤差の範囲で一致している。

\section{検討}

\subsection{水素原子のイオン化エネルギー}

水素原子のイオン化エネルギーは$E = hcR_H$により計算できる。ただし、$h=6.6260755 \times 10^{-34}[{\rm J\cdot s}]$はプランク定数、$c=2.99792458 \times 10^8[{\rm m/s}]$は光速である。
また、Rydberg定数としては測定値$R_H = 1.10\times 10^7[{\rm m^{-1}}]$を用いる。

上式より水素原子のイオン化エネルギーを求めると、

\begin{eqnarray*}
    E & = & hcR_H \\
      & = & 2.19 \times 10^{-18} [{\rm J}] \\
      & = & 13.6 [{\rm eV}]
\end{eqnarray*}

となる。これは、教科書にある値(13.6eV)と一致している。

\subsection{水素原子の大きさの評価}

基底状態の水素の電子の軌道半径は次式で与えられる。

\[
    r = \frac{e^2}{8\pi \epsilon_0 hcR_H}
\]

ただし、素電荷を$e = 1.60217733 \times 10^{-19} {\rm C}$、真空の誘電率を$\epsilon_0 = 8.854188 \times 10^{-12} {\rm F/m}$とする。
この式より基底状態の水素の電子の軌道半径を求めると、

\begin{eqnarray*}
    r & = & \frac{(1.60217733 \times 10^{-19})^2}{8\pi \times (8.854188 \times 10^{-12}) \times (2.19 \times 10^{-18}) } \\
      & = & 5.27 \times 10^{11} [{\rm m}]
\end{eqnarray*}

となる。これは教科書にある値($5.29 \times 10^{-11}$)と一致しないが、$R_H$の値には2\%程度の測定誤差があることから、$r$の値には$0.1 \times 10^{11}$程度の誤差がある。このことを考慮すると、求めた値と教科書にある値が誤差の範囲で一致していることが分かる。

\section{結論}

プリズム分光器を用いて水素原子スペクトルの波長を測定しその値からRydberg定数を求め、それが参照値と一致することを確認した。

\section{感想}

分散曲線などを完全にデジタルな手段で描くことはできないため、一部手動でグラフを書かねばならない局面があり面倒だった。

\end{document}
