\documentclass[uplatex,11pt]{jsarticle}

\usepackage[dvipdfmx]{graphicx}
\usepackage{comment}
%\usepackage{listings, jlisting}
\usepackage{amsmath, amssymb}
\usepackage{wrapfig}
\usepackage{fancybox}
\usepackage{ascmac}
\usepackage{subfig}
\usepackage{rotating}

\setlength\floatsep{5pt}
\setlength\textfloatsep{5pt}
\setlength\intextsep{5pt}
\setlength\abovecaptionskip{0pt}

\newcommand{\fg}[3]{ % \fg{label}{path}{caption},
    \begin{figure}
        \begin{center}
            \includegraphics[width=\textwidth]{#2}
            \caption{#3}
            \label{fg_#1}
        \end{center}
    \end{figure}
}

\newcommand{\dummyfig}[2]{
    \begin{figure}
        \begin{center}
            \begin{shadebox}
                巻末に添付
            \end{shadebox}
            \caption{#2}
            \label{fg_#1}
        \end{center}
    \end{figure}
}

\newcommand{\cir}[3]{ % \cir{label}{path}{caption},
    \begin{figure}
        \begin{center}
            \includegraphics[width=0.85\textwidth]{#2}
            \caption{#3}
            \label{fg_#1}
         \end{center}
    \end{figure}
}

\newcommand{\subtabs}[2]{
    \input{|"ruby ./convert_csv.rb #1 1 '#2'"}
}

\newcommand{\tab}[4]{ % \tab{label}{csv-path}{caption}{分割数}
    \begin{table}[htb]
        \centering
        \caption{#3}
        \input{|"ruby ./convert_csv.rb #2 #4"}
        \label{tab_#1}
    \end{table}
}

\newcommand{\fr}[1]{図\ref{fg_#1}}
\newcommand{\tr}[1]{表\ref{tab_#1}}
\newcommand{\er}[1]{式(\ref{#1})}

\begin{document}

\section{目的}

プリズム分光器を用いて水素原子スペクトルを解析する。

\section{データ}

Na, Hg, Cdの既知のスペクトルに関して、波長とスケールの関係を調べて表にしたのが\tr{data}である。
これをグラフにプロットして分散曲線を描いたのが\fr{bunsan}である。

\tab{data}{csv/bunsan.csv}{実験データ}{2}

また、4本の水素スペクトルを観測してそのスケールを調べたところ以下のようであった。

\begin{itemize}
  \item 1.87
  \item 3.20
  \item 4.05
  \item 4.59
\end{itemize}

\section{解析}

\tr{data}をグラフにプロットして分散曲線を描いたのが\fr{bunsan}である。図から、分散曲線の幅は高々$\Delta\lambda = 5[{\rm nm}]$程度と読み取れる。

\dummyfig{bunsan}{分散曲線}

また、この分散曲線を用いて4本の水素スペクトルの波長を読み取ると以下のようになる。

\begin{itemize}
    \item $662 \pm 5 [{\rm nm}]$
    \item $486 \pm 5 [{\rm nm}]$
    \item $432 \pm 5 [{\rm nm}]$
    \item $410 \pm 5 [{\rm nm}]$
\end{itemize}

\section{考察}



\section{結論}

\end{document}
