\documentclass[uplatex,11pt]{jsarticle}

\usepackage[dvipdfmx]{graphicx}
\usepackage{comment}
%\usepackage{listings, jlisting}
\usepackage{amsmath, amssymb}
\usepackage{wrapfig}
\usepackage{fancybox}
\usepackage{ascmac}
\usepackage{subfig}

\setlength\floatsep{5pt}
\setlength\textfloatsep{5pt}
\setlength\intextsep{5pt}
\setlength\abovecaptionskip{0pt}

\newcommand{\fg}[3]{ % \fg{label}{path}{caption},
    \begin{figure}
        \begin{center}
            \includegraphics[width=\textwidth]{#2}
            \caption{#3}
            \label{fg_#1}
        \end{center}
    \end{figure}
}

\newcommand{\dummyfig}[2]{
    \begin{figure}
        \begin{center}
            \begin{shadebox}
                手書きの図のため、巻末に添付
            \end{shadebox}
            \caption{#2}
            \label{fg_#1}
        \end{center}
    \end{figure}
}

\newcommand{\cir}[3]{ % \cir{label}{path}{caption},
    \begin{figure}
        \begin{center}
            \includegraphics[width=0.85\textwidth]{#2}
            \caption{#3}
            \label{fg_#1}
         \end{center}
    \end{figure}
}

\newcommand{\subtabs}[2]{
    \input{|"LC_CTYPE=C.UTF-8 ruby ./convert_csv.rb #1 1 '#2'"}
}

\newcommand{\tab}[4]{ % \tab{label}{csv-path}{caption}{分割数}
    \begin{table}[htb]
        \centering
        \caption{#3}
        \input{|"LC_CTYPE=C.UTF-8 ruby ./convert_csv.rb #2 #4"}
        \label{tab_#1}
    \end{table}
}

\newcommand{\fr}[1]{図\ref{fg_#1}}
\newcommand{\tr}[1]{表\ref{tab_#1}}
\newcommand{\er}[1]{式(\ref{#1})}

\begin{document}

\section{目的}

平面透過型回折格子を用いて回折角の測定を行い、Na原子のスペクトルD線の波長を求める。

\section{データ}

\tab{data}{csv/data.csv}{各次数の光が現れた角度、またその波長}{1}

実験データとそれから求められる値を表にすると\tr{data}のようになる。

ただし、$m=0,1,2$に関しては、$D_1$の光と$D_2$の光が重なって見えたため$D$の欄を空白としている。

また、次数m、angle(left)、angle(right))の3つは実際に測定した生データであり、それ以外の値は計算によって求めたものである。
計算によって求めた値に関しては、以下のようにして求めた。

\begin{eqnarray*}
    \theta_m         & = & \frac{angle(left) + angle(right)}{2} \\
    \lambda          & = & \frac{\sin\theta_m}{mn} \\
    \Delta \lambda_m & = & \left| \frac{\lambda\Delta\theta}{\tan\theta_m} \right|
\end{eqnarray*}

ただし、$n$は単位長さあたりの刻線数であり、本実験で用いた回折格子では$n = 2.000 \times 10^{6}[m^{-1}]$であった。
また、$\theta$の測定誤差は$\Delta\theta = 2[min] = 5.8 \times 10^{-4} [rad]$とした。これは、実験で用いた分光計は1/2分までの角度を読み取れるが、実際に観測したところ2分程度の違いは判定できなかったからである。


\section{解析}

\fg{data}{graph/a_with_calc.png}{各次数における波長}

\tr{data}について、次数mに対して$\lambda$の大きさを誤差付きでプロットし、教科書より得た$D_1$,$D_2$の文献値である$589.592[nm]$と$588.995[nm]$も書き加えると、\fr{data}のようになる。

\section{考察}

\subsection{平均と加重平均}

$D_1$,$D_2$が観測できた次数について、それぞれ平均値を計算すると、

\begin{eqnarray*}
    D_1(mean) & = & 590.0[nm] \\
    D_2(mean) & = & 589.4[nm]
\end{eqnarray*}

となった。

文献値と$D_1(mean)$と$D_2(mean)$を比較すると、それらの相対誤差はいずれも$0.07\%$となった。

また、$D_1$,$D_2$に関してそれぞれ加重平均を求めると、

\begin{eqnarray*}
    D_1(weighted mean) & = & 589.9[nm] \\
    D_2(weighted mean) & = & 589.3[nm]
\end{eqnarray*}

となり、それらの相対誤差は$0.05\%$となり、単純に平均を取った場合に比べて誤差が小さくなったことが分かる。

\subsection{文献値との差について}

\fr{data}を見ると、$m > 5$において測定値と文献値の差が誤差幅に収まっていないことが分かる。
また、どの$m$についても、測定値が文献値を上回っている。
このことから、この誤差が単なる測定誤差ではなく何らかの系統誤差であると考えられるが、なぜそうなるのか原因はわからなかった。

\section{結論}

測定した波長の値の文献値に対する相対誤差は非常に小さなものとなった。しかし、誤差伝播の法則により求めた誤差幅の中には収まらなかった。

\section{感想}

どうして誤差が出るのかわからず悔しかった。

\end{document}
