\documentclass[uplatex,11pt]{jsarticle}

\usepackage[dvipdfmx]{graphicx}
\usepackage{comment}
%\usepackage{listings, jlisting}
\usepackage{amsmath, amssymb}
\usepackage{wrapfig}
\usepackage{fancybox}
\usepackage{ascmac}
\usepackage{subfig}

\setlength\floatsep{5pt}
\setlength\textfloatsep{5pt}
\setlength\intextsep{5pt}
\setlength\abovecaptionskip{0pt}

\newcommand{\fg}[3]{ % \fg{label}{path}{caption},
    \begin{figure}
        \begin{center}
            \includegraphics[width=\textwidth]{#2}
            \caption{#3}
            \label{fg_#1}
        \end{center}
    \end{figure}
}

\newcommand{\dummyfig}[2]{
    \begin{figure}
        \begin{center}
            \begin{shadebox}
                手書きの図のため、巻末に添付
            \end{shadebox}
            \caption{#2}
            \label{fg_#1}
        \end{center}
    \end{figure}
}

\newcommand{\cir}[3]{ % \cir{label}{path}{caption},
    \begin{figure}
        \begin{center}
            \includegraphics[width=0.85\textwidth]{#2}
            \caption{#3}
            \label{fg_#1}
         \end{center}
    \end{figure}
}

\newcommand{\subtabs}[2]{
    \input{|"ruby ./convert_csv.rb #1 1 '#2'"}
}

\newcommand{\tab}[4]{ % \tab{label}{csv-path}{caption}{分割数}
    \begin{table}[htb]
        \centering
        \caption{#3}
        \input{|"ruby ./convert_csv.rb #2 #4"}
        \label{tab_#1}
    \end{table}
}

\newcommand{\fr}[1]{図\ref{fg_#1}}
\newcommand{\tr}[1]{表\ref{tab_#1}}
\newcommand{\er}[1]{式(\ref{#1})}

\begin{document}

\section{目的}

本実験の目的は、原子が離散的なエネルギー状態を持つことを電子ビームとNe原子の非弾性衝突の実験によって確認し、Ne原子の最低励起エネルギーを求めることである。

\section{原理}

\subsection{背景}

\subsubsection{Bohrの量子仮説}

Bohrの量子仮説によると、原子にはとびとびのエネルギー値を持った状態(定常状態)のみが許され、定常状態から定常状態に遷移するときのみエネルギーの変化が起きる。また、状態$n$から$m$への遷移が起こる時に放出または吸収される電磁波の振動数$\mu$は、プランク定数を$h$とすると、

\begin{eqnarray*}
    h\mu = \left|E_n - E_m\right|
\end{eqnarray*}

を満たす。

\subsubsection{フランク・ヘルツの実験}

電子を加速電圧$V_{ACC}$によって加速することを考えると、その電子の運動エネルギーは電子ボルト単位${\rm eV}$で$eV_{ACC}[{\rm eV}]$と表される。ただし、$e$は電気素量である。
この電子が原子に衝突した時、その運動エネルギーが原子の最低励起エネルギーを超えていれば非弾性衝突が起こり、原子のエネルギー状態が遷移する。
この時、その原子の基底状態のエネルギーを$eV_0[{\rm eV}]$、そのすぐ上の励起状態のエネルギーを$eV_1[{\rm eV}]$とすると、その原子の最低励起エネルギー$V_E$は、

\begin{eqnarray*}
    V_E = V_1 - V_0
\end{eqnarray*}

となる。

\subsection{測定原理}

\subsubsection{フランク・ヘルツの実験}

\dummyfig{vaccum_tube}{フランク・ヘルツの実験装置の原理図}

\fr{vaccum_tube}のような装置を用いる。
カソードKをヒーターによって加熱し、Neガスを封入した4極管にて電子を放出する。
放出された電子をグリッド$G_1$、$G_2$の間で加速する。

加速電圧$V_{ACC}$を0Vから増加させると、最初、電子はNe原子と衝突しても非弾性衝突を起こさない。
そのため電子はエネルギーを失わずプレートPに到達する。
このとき、カソードKから流れる電流は熱電子管の電圧-電流特性で決まり、単調に増加する。
ただし、グリッド$G_2$とプレートPの間には約6Vの逆電圧が印加されているため、プレート電流$I_P$は、$V_{ACC}$が約6Vになるあたりから増加し始める。

さらに$V_{ACC}$を増加させ電子のエネルギーが$eV_E$付近になると電子とNe原子の非弾性衝突が起こりはじめ、電子のエネルギーが失われる。
プレートPには逆電圧が印加されているため、逆電圧に対応するエネルギー以下になった電子はプレートPに到達できず、$I_P$は急激に減少する。
さらに$V_{ACC}$を増加させると、非弾性衝突をしたあとの電子のエネルギーが逆電圧に対応するエネルギーを超え、再び$I_P$は増大する。

これを繰り返すことにより、$V_{ACC}$が$V_E$だけ増加するごとに$I_P$は極大値を示す。

\subsubsection{励起発光の観測}

Neの最低励起エネルギーは16.7Vであり、対応する光は紫外線領域である。従って、人間の目でこれを観測することはできない。
しかし、実験中に4極管を覗くと赤色系の光を観測することができる。
この光は、Ne原子の価電子1個が3p軌道から3s軌道に遷移する際に放出される光(波長585.2488nm)であると考えられ、そのエネルギーは2.11eVである。

このことは、Ne原子が基底状態から直接3pなどへ励起されるか、一旦3sに励起されたNe原子が続けて3pなどに励起され、その後3pから3sへと発行を伴った遷移が発生したことを示唆している。

\section{装置}

\begin{enumerate}
    \item Franck-Hertz実験装置(京都大学-KKエクレア社協同制作)
    \item 0〜100V 直流電圧計($V_{ACC}$測定用)
    \item 0〜1A 直流電流計(ヒータ電流$I_H$測定用)
    \item 0〜100$\mu {\rm A}$ 直流電流計($I_P$測定用)
\end{enumerate}

Franck-Hertz実験装置の詳細な説明は割愛する。

\section{方法}

\subsection{測定準備}

Franck-Hertz実験装置に3つの電流計と電圧計を接続した。
パネル上面のヒータ電流用ダイヤルと加速電圧用のダイヤルが左いっぱいに回っていることを確認した後、本体の電源コードをコンセントに差し込んだ。
その後POWERスイッチをONにし、ヒータ電流が400mA程度流れていることを確認し、カソードの温度が十分高くなるまで1分程度ウォーミングアップした。
その後、PLATE CURRENTの零点を零点調整用ダイヤルを用いて合わせた。

$V_{ACC}$を約90Vに設定し、ヒータ電流を約600mA前後までゆっくりと上げ、プレート電流$I_P$が60〜80$\mu {\rm A}$になるように調節した。
そして、$I_P$の変動が小さくなるまで再度ウォーミングアップした。
この後、ヒータ電流の値を記録して加速電圧を0Vにし、再度$\mu {\rm A}$計の零点調整を行った。

\subsection{プレート電流の加速電圧依存性の測定}

加速電圧$V_{ACC}$を0Vからゆっくりと上げながら、プレート電流$I_P$が4〜5個のピークを持つことを確かめた。
このとき、ピークに対応する加速電圧の値を記録した。

次に、$V_{ACC}$を0Vから80V付近まで少しづつ上げ、$V_{ACC}$とそのときの$I_P$の値を記録した。
このとき、$I_P$の極大値や極小値付近では0.2〜0.5V間隔で$V_{ACC}$を少しづつ変化させ、それ以外の部分では1V間隔で大きく変化させた。

上記の手順をヒータ電流の値と測定者を変えて2回行った。

\subsection{励起発光の観察}

2本の遮光筒を接続して真空管の観測窓から差込み、加速電圧$V_{ACC}$を変化させて同心円状の励起発行のリングパターンを観測した。
リングが現れる度にその時の$V_{ACC}$の値を記録した。
また、励起発光のリングパターンの個数や半径が、$V_{ACC}$の値によってどのように変化したかを記録した。

\section{データ}
\section{解析}
\section{考察}
\section{結論}

\end{document}
