\documentclass[uplatex,11pt]{jsarticle}

\usepackage[dvipdfmx]{graphicx}
\usepackage{comment}
%\usepackage{listings, jlisting}
\usepackage{amsmath, amssymb}
\usepackage{wrapfig}
\usepackage{fancybox}
\usepackage{ascmac}
\usepackage{subfig}
\usepackage{url}

\setlength\floatsep{5pt}
\setlength\textfloatsep{5pt}
\setlength\intextsep{5pt}
\setlength\abovecaptionskip{0pt}

\newcommand{\fg}[3]{ % \fg{label}{path}{caption},
    \begin{figure}
        \begin{center}
            \includegraphics[width=\textwidth]{#2}
            \caption{#3}
            \label{fg_#1}
        \end{center}
    \end{figure}
}

\newcommand{\dummyfig}[2]{
    \begin{figure}
        \begin{center}
            \begin{shadebox}
                手書きの図のため、巻末に添付
            \end{shadebox}
            \caption{#2}
            \label{fg_#1}
        \end{center}
    \end{figure}
}

\newcommand{\cir}[3]{ % \cir{label}{path}{caption},
    \begin{figure}
        \begin{center}
            \includegraphics[width=0.85\textwidth]{#2}
            \caption{#3}
            \label{fg_#1}
         \end{center}
    \end{figure}
}

\newcommand{\subtabs}[2]{
    \input{|"ruby ./convert_csv.rb #1 1 '#2'"}
}

\newcommand{\tab}[4]{ % \tab{label}{csv-path}{caption}{分割数}
    \begin{table}[htb]
        \centering
        \caption{#3}
        \input{|"ruby ./convert_csv.rb #2 #4"}
        \label{tab_#1}
    \end{table}
}

\newcommand{\fr}[1]{図\ref{fg_#1}}
\newcommand{\tr}[1]{表\ref{tab_#1}}
\newcommand{\er}[1]{式(\ref{#1})}

\begin{document}

\section{目的}

磁場の大きさを測定するためにホール素子の較正をし、それを用いて有限の長さのソレノイドコイルが作る磁場の様子を調べる。

\section{原理}

\subsection{ホール効果}

直流電流$I_H$があるとき、これに直交する磁場$H$を加えると、$I_H$を構成する電子は$I_H$と$H$の両者に垂直な向きにローレンツ力を受ける。
電流$I_H$を構成する電子の分布は、このローレンツ力によって偏るが、その偏りによって生じた電場によって力を受ける。
そして、電子がローレンツ力と電場から受ける力とが釣り合い定常状態になり、電圧が生じる。
これをホール効果という。

定常状態において力の向きに生じるホール電圧$V_H$は、$I_H$と$H$の大きさに比例する。
この関係は比例係数$a$を用いて、

\begin{eqnarray}
    V_H = a I_H H \label{eq_vh_Ih_H_relation}
\end{eqnarray}

のように表される。

このことから、既知の磁場と電流で$V_H$を測定し比例係数$a$を求めることで、$V_H$の値から磁場の大きさを測定することができる。

\subsection{有限の長さのソレノイドコイル}

円電流の作る磁場について考える。
円電流の中心軸上の、円電流の中心から距離$x$にある点$P$での磁場の大きさ$H_1$は、ビオ・サバールの法則から、

\[
    H_1(x) = \frac{Ia^2}{2(a^2+x^2)^\frac{3}{2}}
\]

となる。ただし、$a$は円電流の半径、$I$は円電流の大きさである。

これを用いて有限の長さのソレノイドコイルの中心軸上の点における磁場の大きさ$H$を求める。
ソレノイドコイルを円電流が複数重なったものと考える。
ソレノイドコイルの中心から距離$x$の点$Q$における磁場の強さは、

\begin{eqnarray}
    H & = & \int_{x + l}^{x - l} H_1(x^{\prime}) ndx^{\prime} \nonumber \\
      & = & \frac{nI}{2} \left[ \frac{x + l}{\sqrt{a^2+(x+l)^2}} - \frac{x - l}{\sqrt{a^2+(x-l)^2}} \right] \label{eq_solenoid_theoretical}
\end{eqnarray}

と表される。ただし、$n$は単位長さあたりのコイルの巻数、$l$はソレノイドコイルの長さ、$I$はソレノイドコイルに流れる電流の大きさである。

\section{装置}

\begin{description}
    \item[ホール素子] TOSHIBA THS118, GaAs
    \item[ホール素子にかける電源] KIKUSUI PMX500 - 0.1A
    \item[ホール素子の電圧計] KEITHLEY 2110 5 $\frac{1}{2}$ Digit Multimeter 
    \item[ソレノイドコイル用の電源] MATASUSADA
    \item[ソレノイドコイル] $n = 20000 m^{-1}$, $l = 0.15 m$, $a = 2.5 cm$
\end{description}

ただし、ソレノイドコイルの半径$a$に関しては目測による値である。

\section{方法}

\subsection{ホール素子の較正}

ホール素子を電流源と電圧計に接続し、ソレノイドコイルの中心にホール素子を設置した。

その後、以下の3つの場合についてホール電圧$V_H$、ホール電流$I_H$、ソレノイドコイルの電流$I_{mag}$を測定した

\subsubsection{磁場を印加しない場合}

ソレノイドコイルに電流を流さずに、$I_H$を0から10mAまで1mA刻みで変化させたときに発生するホール電圧$V_H$を測定した。

\subsubsection{磁場が一定の場合}

ソレノイドコイルに一定の電流($I_{mag} = 1.0[A]$)を流し、$I_H$を0から10mAまで1mA刻みで変化させたときに発生するホール電圧$V_H$を測定した。

\subsubsection{磁場を変化させた場合}

$I_H$を一定の大きさ($I_H = 10mA$)に固定してからソレノイドコイルの電流を0にし、電圧計の[null]ボタンを押してオフセットを設定した。

その状態で、ソレノイドコイルに流す電流を0から1.0Aまで0.1A刻みで変化させたときに発生するホール電圧$V_H$を測定した。

\subsection{ソレノイドコイルの磁場の測定}

\subsubsection{中心軸に沿った磁場の分布}

$I_H = 10mA$、$I_{mag} = 1.0A$で固定し、ホール素子をソレノイドコイルの中心に設置した。

その後、ホール素子をソレノイドコイルの中心軸(x軸)に沿って移動させていき、中心からの距離$x[m]$における$V_H$を測定した。

\subsubsection{中心軸と垂直な軸に沿った磁場の分布}
ソレノイドコイルの開口部近傍で、ホール素子を中心軸と垂直な軸(y軸)に沿って動かし、その位置$y[m]$における$V_H$を測定した。

\section{結果}

\subsection{ホール素子の較正}

ホール素子の較正の実験を行った結果が\tr{a_2}、\tr{a_3}、\tr{a_4}である。

\subsection{ソレノイドコイルの磁場の測定}

ソレノイドコイルの磁場の測定の実験を行った結果が\tr{b_1}、\tr{b_2}である。

\section{解析}
\subsection{ホール素子の較正}

\tr{a_2}、\tr{a_3}、\tr{a_4}をグラフにしたものが\fr{a_2}、\fr{a_3}、\fr{a_4}である。ただし、\fr{a_4}に関しては、$H = (2 \times 10^4) \times I_{mag}$によって$I_{mag}$を$H$に変換している。

どのグラフに関しても、原点を通る直線であることが分かる。また、最小二乗法によって、\fr{a_4}の直線の傾きは$415$であるとわかる。このことから、$I_H = 10[mA]$の条件下では、

\begin{eqnarray}
    H = 415 \times V_H \label{eq_vh_to_h}
\end{eqnarray}

であることがわかる。ただし、上式中の$V_H$はmVスケールの値である。

\subsection{ソレノイドコイルの磁場の測定}

\tr{b_1}、\tr{b_2}をグラフにしたものが\fr{b_1}、\fr{b_2}である。ただし、グラフ中の$H$は、測定した$V_H$の値を\er{eq_vh_to_h}によって$H$に変換したものである。また、\fr{b_2}については、$V_H = 0$となる点である$y_0=0.21m$を中心としてグラフを描いた。

また、$ \frac{dH}{dx} $と$ \frac{dH}{dy} $を、

\[
    \frac{dH}{dx} \approx \frac{\Delta H}{ \Delta x }
\]

によって近似的に求め、グラフにプロットしたものが\fr{b_1_differential}、\fr{b_2_differential}である。
このときも、$y_0=0.21m$を中心としてグラフを描いた。

また、\er{eq_solenoid_theoretical}から、x軸に沿った磁場の分布の理論値を求め、実験結果を重ねてグラフを描いたものが\fr{b_1_theoretical_comparizon}であり、その相対誤差をグラフにしたものが\fr{b_1_relative_error}で、絶対誤差をグラフにしたものが\fr{b_1_absolute_error}である。

\section{考察}

\subsection{ホール電圧の特性について}

$V_H$と$I_H$、$V_H$と$H$の関係は、\fr{a_2}、\fr{a_3}、\fr{a_4}を見ると比例関係であることがわかる。
この結果は \er{eq_vh_Ih_H_relation} と一致する。

\subsection{x軸上の磁場分布について}

\fr{b_1_theoretical_comparizon}を見ると実験結果と理論値は概ね一致しているように見えるが、その相対誤差を取ると(\fr{b_1_relative_error})、xが増大するに連れて急激に増大していることが分かる。

ここで、絶対誤差のグラフ\fr{b_1_absolute_error}を見ると、xが大きい場所では$H \approx 124[A/m]$でほぼ一定となっていることが分かる。
これより、実験結果の磁場の大きさから$ H = 124[A/m] $を引いた上で理論値との相対誤差を取り、グラフにしたのが\fr{b_1_relative_error_with_offset}である。

\fr{b_1_relative_error_with_offset}を見ると、高々10\%程度の誤差に収まっていることがわかる。
また、高めの誤差が出ている場所は、$x = 0.2[m]$付近と$x = 0.4[m]$付近の二箇所である。
前者に関しては、xに対してHが急激に変化している領域であり、xの誤差にHが過敏に反応しているものと考えられる。
後者に関しては、H自体が非常に小さな値であるため、測定誤差が大きく影響しているものだと考えられる。

では、$ H = 124[A/m] $とは何か。

まず考えられるのが、地磁気の影響である。理科年表オフィシャルサイトの記述から、京都での地磁気の大きさは$5 \times 10^{-5} [T] = 0.5 [G] = 40 [A/m]$程度であると推定できる。この値は、$H = 124[A/m]$と概ねオーダーが一致している。

また、他の原因として考えられるのは、$V_H$を測定する端子が僅かにずれてしまい、$V_H$だけでなく$I_H$による電圧降下も測定してしまっているという可能性である。
これに関しては、ホール素子の形状や素材が詳しくわからないため評価することができない。

\subsection{磁場の発散}

ソレノイドコイルの口部分における磁場の発散${\rm div} {\bf H}$を求める。

磁場のxz面での対称性より$\frac{\partial H_y}{\partial y} = \frac{\partial H_z}{\partial z}$とし、また$\frac{\partial H_x}{\partial x} = \frac{\Delta H_x}{\Delta x}$と近似すると、

\begin{eqnarray*}
    {\rm div} {\bf H} & = & \frac{\partial H_x}{\partial x} + \frac{\partial H_y}{\partial y} + \frac{\partial H_z}{\partial z} \\
                      & \approx & \frac{\Delta H_x}{\Delta x} + 2\frac{\Delta H_y}{\Delta y}
\end{eqnarray*}

となる。
これより、$(x, y) = (0.145[m], 0.210[m])$である点(コイルの口部分)に関して${\rm div} {\bf H}$を求めると、

\begin{eqnarray*}
    {\rm div} {\bf H} & = & \frac{\Delta H_x}{\Delta x} + 2\frac{\Delta H_y}{\Delta y} \\
                      & = & \frac{9.67 \times 10^3 - 1.18 \times 10^4}{0.150 - 0.145} + 2\frac{3.40 \times 10 ^2 - (-5.18 \times 10)}{0.215 - 0.210} \\
                      & = & -4.31 \times 10^5 + 2 \times 7.84 \times 10 ^ 4 \\
                      & = & - 2.74 \times 10^5
\end{eqnarray*}

となる。磁場は湧き出しが存在しないので${\rm div} {\bf H} = 0$となるはずであり、上式の結果と一致しないことが分かる。

この原因として考えられるのは、「x軸上で磁場の値を計測した地点」と「y軸上で磁場の値を計測した地点」にズレがあるということである。つまり、$x=0.145[m]$である地点のyの値が$y=0.210[m]$でないということである。というのも、実験器具の形状から、y軸上の磁場を測定する際、ホール素子をコイルの口に密着させる事ができず、少し口から離れた状態であったからである。
また、\fr{b_1_differential}や\fr{b_2_differential}を見ると分かる通り、$(x, y) = (0.145, 0.210)$付近の点は、$\frac{\Delta H_x}{\Delta x}$、$\frac{\Delta H_y}{\Delta y}$ともに変化が激しいため、少し位置がズレただけで${\rm div} {\bf H}$の値は大きく変化する。

ところで、$y=0.210[m]$として、xの値を動かして${\rm div} {\bf H}$のグラフを描くと、\fr{x_divH}のようになり、$x=0.17[m]$周辺で${\rm div} {\bf H} = 0$となることがわかる。
このことから、実際にx軸とy軸が交わっていたのは、$(x, y) = (0.17, 0.21)$付近であったと推定できる。

\subsection{x軸上の磁場分布からコイルの総巻数を推定する}

ソレノイドコイルにアンペールの法則

\[
    \oint_C {\bf H}_s d{\bf s} = NI
\]

を適用することで、コイルの総巻数$N$を推定する。

積分経路$C$として、コイルのx軸上を通り地球をぐるっと回って逆側の口からコイルに入るような経路を考える。
実験結果の範囲では$x$が一定以上( $ x > 0.4 $ )になると$H$はほとんど0になっていることから、上記の経路での周回積分を求めるためには、$\int^{0.4}_{-0.4} H dx$を求めれば良いとわかる。
また、ソレノイドコイルの磁場が、$x=0$について対称であると仮定すると、

\[
\int^{0.4}_{-0.4} H dx = 2 \int^{0.4}_0 H dx
\]

である。ここで、実験結果から台形法によって近似的に$\int^{0.4}_0 H dx$を求めると、$3.0 \times 10 ^ 3 $程度であった。
このことから、$NI = 6.0 \times 10^3$であるとわかる。そして、実験時は$I = 1.0[A]$であったことから、$N = 6.0 \times 10^3$と求められる。

教科書によると$N = 6000$であり、一致していることが分かる。

\section{結論}

ホール素子の特性が、\er{eq_vh_Ih_H_relation}に従っていることを確かめた。

また、有限の長さのソレノイドコイルが作る磁場の分布が\er{eq_solenoid_theoretical}に従うことを確かめた。
さらに、ソレノイドコイルが作る磁場が、アンペールの法則に従うことを確かめた。

\section{参考文献}

\begin{itemize}
    \item 理科年表オフィシャルサイト 地磁気活動度指数 \url{https://www.rikanenpyo.jp/kaisetsu/chigaku/chigaku_004.html}
\end{itemize}


\newpage
\section*{図表}

\tab{a_2}{./csv/jikken_a_2.csv}{ $H=0[A/m]$の元での$I_H$と$V_H$の関係 }{1}
\tab{a_3}{./csv/jikken_a_3.csv}{ $H=20000[A/m]$の元での$I_H$と$V_H$の関係 }{1}
\tab{a_4}{./csv/jikken_a_4.csv}{ $I_H = 10[mA]$の元での$I_{mag}$と$V_H$の関係 }{1}

\tab{b_1}{./csv/jikken_b_1.csv}{x軸に沿った磁場の分布}{3}
\tab{b_2}{./csv/jikken_b_2.csv}{y軸に沿った磁場の分布}{3}

\clearpage

\cir{a_2}{./graph/jikken_a_2.png}{ $ H=0[A/m] $の元での$I_H$と$V_H$の関係 }
\cir{a_3}{./graph/jikken_a_3.png}{ $ H=20000[A/m] $の元での$I_H$と$V_H$の関係 }
\cir{a_4}{./graph/jikken_a_4.png}{ $ I_H = 10[mA] $の元での$H$と$V_H$の関係 }

\cir{b_1}{./graph/jikken_b_1.png}{ x軸に沿った磁場の分布 }
\cir{b_2}{./graph/jikken_b_2.png}{ y軸に沿った磁場の分布(ただし、y=0.21mを中心とした) }

\cir{b_1_differential}{./graph/jikken_b_1_differential.png}{ $\frac{dH}{dx}$の分布 }
\cir{b_2_differential}{./graph/jikken_b_2_differential.png}{ $\frac{dH}{dy}$の分布(ただし、y=0.21mを中心とした) }

\cir{b_1_theoretical_comparizon}{./graph/jikken_b_1_theoretical_comparizon.png}{ x軸に沿った磁場の分布とその理論値 }
\cir{b_1_relative_error}{./graph/jikken_b_1_relative_error.png}{ x軸に沿った磁場の分布の相対誤差 }
\cir{b_1_absolute_error}{./graph/jikken_b_1_absolute_error.png}{ x軸に沿った磁場の分布の絶対誤差 }

\cir{b_1_relative_error_with_offset}{./graph/jikken_b_1_relative_error_with_offset.png}{ x軸に沿った磁場の分布の相対誤差(Hから一定値を引いたもの) }

\cir{x_divH}{./graph/x_divH.png}{ xに対する${\rm div} {\bf H}$の値 }

\end{document}
