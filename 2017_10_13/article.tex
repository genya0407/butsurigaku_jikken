\documentclass[uplatex,11pt]{jsarticle}

\usepackage[dvipdfmx]{graphicx}
\usepackage{comment}
%\usepackage{listings, jlisting}
\usepackage{amsmath, amssymb}
\usepackage{wrapfig}
\usepackage{fancybox}
\usepackage{ascmac}
\usepackage{subfig}

\setlength\floatsep{5pt}
\setlength\textfloatsep{5pt}
\setlength\intextsep{5pt}
\setlength\abovecaptionskip{0pt}

\newcommand{\fg}[3]{ % \fg{label}{path}{caption},
    \begin{figure}
        \begin{center}
            \includegraphics[width=\textwidth]{./figs/#2}
            \caption{#3}
            \label{#1}
        \end{center}
    \end{figure}
}

\newcommand{\dummyfig}[2]{
    \begin{figure}
        \begin{center}
            \begin{shadebox}
                手書きの図のため、巻末に添付
            \end{shadebox}
            \caption{#2}
            \label{#1}
        \end{center}
    \end{figure}
}

\newcommand{\cir}[3]{ % \cir{label}{path}{caption},
    \begin{figure}
        \begin{center}
            \includegraphics[width=0.7\textwidth]{./figs/#2}
            \caption{#3}
            \label{#1}
         \end{center}
    \end{figure}
}

\newcommand{\subtabs}[2]{
    \input{|"ruby ./convert_csv.rb #1 1 '#2'"}
}

\newcommand{\tab}[4]{ % \tab{label}{csv-path}{caption}{分割数}
    \begin{table}[htb]
        \centering
        \input{|"ruby ./convert_csv.rb #2 #4"}
        \caption{#3}
        \label{#1}
    \end{table}
}

\newcommand{\fr}[1]{図\ref{#1}}
\newcommand{\tr}[1]{表\ref{#1}}
\newcommand{\er}[1]{式(\ref{#1})}

\begin{document}

\section{目的}

有限の長さのソレノイドコイルが作る磁場の様子を調べる。
また、磁場の測定に用いるホール素子の較正をする。

\section{原理}

\subsection{ホール効果}

直流電流$I_H$があるとき、これに直交する磁場$H$を加えると、$I_H$を構成する電子は$I_H$、$H$に垂直な向きにローレンツ力を受ける。
電流$I_H$を構成する電子の分布は、このローレンツ力によって偏るが、その偏りによって生じた電場によって力を受ける。
そして、ローレンツ力と電場から受ける力とが釣り合い定常状態になり、電圧が生じる。
これをホール効果という。

定常状態において力の向きに生じるホール電圧$V_H$は、$I_H$と$H$の大きさに比例する。
この関係は比例係数$a$を用いて、

\[
    V_H = a I_H H
\]

のように表される。

このことから、既知の磁場と電流で$V_H$を測定し比例係数$a$を求めることで、$V_H$の値から磁場の大きさを測定することができる。

\subsection{有限の長さのソレノイドコイル}

円電流の作る磁場について考える。
円電流の中心軸上の、円電流の中心から距離$x$にある点$P$での磁場の大きさ$H_1$は、ビオ・サバールの法則から、

\[
    H_1(x) = \frac{Ia^2}{2(a^2+x^2)^\frac{3}{2}}
\]

となる。ただし、$a$は円電流の半径、$I$は円電流の大きさである。

これを用いて有限の長さのソレノイドコイルの中心軸上の点における磁場の大きさ$H$を求める。
ソレノイドコイルを円電流が複数重なったものと考える。
ソレノイドコイルの中心から距離$x$の点$Q$における磁場の強さは、

\begin{eqnarray}
    H & = & \int_{x + l}^{x - l} H_1(x^{\prime}) ndx^{\prime} \\
      & = & \frac{nI}{2} \left( \frac{x + l}{} \right)
\end{eqnarray}

と表される。ただし、$n$は単位長さあたりのコイルの巻数、$l$はソレノイドコイルの長さ、$I$はソレノイドコイルに流れる電流の大きさである。

\section{装置}

\begin{description}
    \item[ホール素子] TOSHIBA THS118, GaAs
    \item[ホール素子にかける電源] KIKUSUI PMX500 - 0.1A
    \item[ホール素子の電圧計] KEITHLEY 2110 5 $\frac{1}{2}$ Digit Multimeter 
    \item[ソレノイドコイル用の電源] MATASUSADA
    \item[ソレノイドコイル] $n = 20000 m^{-1}$, $l = 0.15 m$, $a = 2.5 cm$
\end{description}

ただし、ソレノイドコイルの半径$a$に関しては目測による値である。

\section{方法}

\subsection{ホール素子の較正}

ホール素子を電流源と電圧計に接続し、ソレノイドコイルの中心にホール素子を設置した。

その後、以下の3つの場合についてホール電圧$V_H$、ホール電流$I_H$、ソレノイドコイルの電流$I_mag$を測定した

\subsubsection{磁場を印加しない場合}

ソレノイドコイルに電流を流さずに、$I_H$を0から10mAまで1mA刻みで変化させたときに発生するホール電圧$V_H$を測定した。

\subsubsection{磁場が一定の場合}

ソレノイドコイルに一定の電流($I_mag = 1.0A$)を流し、$I_H$を0から10mAまで1mA刻みで変化させたときに発生するホール電圧$V_H$を測定した。

\subsubsection{磁場を変化させた場合}

$I_H$を一定の大きさ($I_H = 10mA$)に固定してからソレノイドコイルの電流を0にし、電圧計の[null]ボタンを押してオフセットを設定した。

その状態で、ソレノイドコイルに流す電流を0から1.0Aまで0.1A刻みで変化させたときに発生するホール電圧$V_H$を測定した。

\subsection{ソレノイドコイルの磁場の測定}

\subsubsection{中心軸に沿った磁場の分布}

$I_H = 10mA$、$I_mag = 1.0A$で固定し、ホール素子をソレノイドコイルの中心に設置した。

その後、ホール素子をソレノイドコイルの中心軸(x軸)に沿って移動させていき、中心からの距離$x[m]$における$V_H$を測定した。

\subsubsection{中心軸と垂直な軸に沿った磁場の分布}
ソレノイドコイルの開口部近傍で、ホール素子を中心軸と垂直な軸(y軸)に沿って動かし、その位置$y[m]$における$V_H$を測定した。

\section{データ}
\subsection{ホール素子の較正}

\tab{a_2}{./csv/jikken_a_2.csv}{ $H=0[A/m]$の元での$I_H$と$V_H$の関係 }{1}
\tab{a_3}{./csv/jikken_a_3.csv}{ $H=20000[A/m]$の元での$I_H$と$V_H$の関係 }{1}
\tab{3_4}{./csv/jikken_a_4.csv}{ $I_H = 10[mA]$の元での$I_{mag}$と$V_H$の関係 }{1}

\subsection{ソレノイドコイルの磁場の測定}

\tab{b_1}{./csv/jikken_b_1.csv}{x軸に沿った磁場の分布}{3}
\tab{b_2}{./csv/jikken_b_2.csv}{y軸に沿った磁場の分布}{3}

\section{解析}
% グラフ書いたり
\subsection{ホール素子の較正}

\subsection{ソレノイドコイルの磁場の測定}

\section{考察}
\section{結論}

\end{document}
