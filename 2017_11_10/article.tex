\documentclass[uplatex,11pt]{jsarticle}

\usepackage[dvipdfmx]{graphicx}
\usepackage{comment}
%\usepackage{listings, jlisting}
\usepackage{amsmath, amssymb}
\usepackage{wrapfig}
\usepackage{fancybox}
\usepackage{ascmac}
\usepackage{subfig}

\setlength\floatsep{5pt}
\setlength\textfloatsep{5pt}
\setlength\intextsep{5pt}
\setlength\abovecaptionskip{0pt}

\newcommand{\fg}[3]{ % \fg{label}{path}{caption},
    \begin{figure}
        \begin{center}
            \includegraphics[width=\textwidth]{#2}
            \caption{#3}
            \label{fg_#1}
        \end{center}
    \end{figure}
}

\newcommand{\dummyfig}[2]{
    \begin{figure}
        \begin{center}
            \begin{shadebox}
                手書きの図のため、巻末に添付
            \end{shadebox}
            \caption{#2}
            \label{fg_#1}
        \end{center}
    \end{figure}
}

\newcommand{\cir}[3]{ % \cir{label}{path}{caption},
    \begin{figure}
        \begin{center}
            \includegraphics[width=0.85\textwidth]{#2}
            \caption{#3}
            \label{fg_#1}
         \end{center}
    \end{figure}
}

\newcommand{\subtabs}[2]{
    \input{|"ruby ./convert_csv.rb #1 1 '#2'"}
}

\newcommand{\tab}[4]{ % \tab{label}{csv-path}{caption}{分割数}
    \begin{table}[htb]
        \centering
        \caption{#3}
        \input{|"ruby ./convert_csv.rb #2 #4"}
        \label{tab_#1}
    \end{table}
}

\newcommand{\fr}[1]{図\ref{fg_#1}}
\newcommand{\tr}[1]{表\ref{tab_#1}}
\newcommand{\er}[1]{式(\ref{#1})}

\begin{document}

\section{目的}

連成振動のモデルの検証を行なう。

\section{原理}

\subsection*{単振子の特性}

回転軸から振動子の重心までの距離をh[m]、振動子の質量をm[kg]、回転軸に対する振動子の慣性モーメントをI[kg${\rm m^2}$]とすると、振動子の振れの角度$\theta$は、

\begin{eqnarray*}
    I\frac{d^2\theta}{dt^2} & = & -mgh \sin \theta \\
                      & \approx & -mgh \theta
\end{eqnarray*}

と表される。このことから、単振子の振動数f[Hz]は、

\[
    f = \frac{1}{2\pi}\sqrt{\frac{mgh}{I}}
\]

と求まる。

\subsection*{回転軸の結合係数}

一つの回転軸に同じ質量で同じ腕の長さの2つの振子(A, B)を同じ向きに固定して土台に置き、重力に従って垂れさせる。
この時、片方の振子に力を加えて$\theta_A$だけ回転させて固定すると、もう片方の振子も回転軸から加わるトルクNによって回転する。
そして、$\theta_B$だけ回転したところで重力のトルク$mgh\sin\theta_B$とNが釣り合って停止する。

ところで、ねじれの角度$\Delta \theta = \theta_A - \theta_B$が$\pi$に比べて十分小さいとき、Nと$\Delta \theta$の間には比例関係があると考えて良い。
このことから、結合定数cを導入することによってトルクの釣り合いの式は

\[
    mgh\sin\theta_B = c(\theta_A - \theta_B)
\]

と表されることが分かる。

\subsection*{連成振動の特性}

\subsubsection*{連成振子の動き}

一つの回転軸に同じ質量で同じ腕の長さの2つの振子(A, B)を固定して振動させることを考える。
このとき、振子A, Bはそれぞれ重力のモーメントと回転軸のねじれから生じる復元力のモーメントが働く。

このことから運動方程式を立てると、

\begin{eqnarray*}
    I\frac{d^2\theta_A}{dt^2} & = & -mgh \sin \theta_A + c(\theta_B - \theta_A) \approx -mgh\theta_A  + c(\theta_B - \theta_A) \\
    I\frac{d^2\theta_B}{dt^2} & = & -mgh \sin \theta_B - c(\theta_B - \theta_A) \approx -mgh\theta_B  - c(\theta_B - \theta_A)
\end{eqnarray*}

となる。これを、変数変換、

\begin{eqnarray*}
    \theta_G & = & \frac{\theta_A + \theta_B}{2} \\
    \theta_R & = & \frac{\theta_A - \theta_B}{2}
\end{eqnarray*}

を用いて解くと、

\begin{eqnarray*}
    \theta_G & = & A\cos(\omega_1t + \phi_1) \\
    \theta_R & = & B\cos(\omega_2t + \phi_2)
\end{eqnarray*}

となる。ただし、$\omega_1 = \sqrt{mgh/I}$、$\omega_2 = \sqrt{(mgh+2c)/I}$であり、A、B、$\phi_1$、$\phi_2$は未定定数である。

ここで、$\theta_G$、$\theta_R$から$\theta_A$、$\theta_B$を求めると、

\begin{eqnarray*}
    \theta_A & = & A\cos(\omega_1t + \phi_1) + B\cos(\omega_2t + \phi_2) \\
    \theta_B & = & A\cos(\omega_1t + \phi_1) - B\cos(\omega_2t + \phi_2)
\end{eqnarray*}

が得られる。このことから、2つの結合した振子の運動は基準振動の重ね合わせで表せることが分かる。

\subsubsection*{基準振動}

以下の2つの初期条件について振動数を測定することで、基準振動の角振動数$\omega_1$、$\omega_2$を実験的に得ることができる。

まず、時刻$t = 0$において$\theta_A = \theta_B = \theta_0$で振子が静止している場合について考えると、

\[
    \theta_A = \theta_0\cos(\omega_1 t), \theta_B = \theta_0 \cos(\omega_1 t)
\]

が成り立つ。従って、振子A,Bの振動の周波数を計測することで$\omega_1$を求めることができる。

次に、時刻$t=0$において$\theta_A = \theta_0$, $\theta_b = - \theta_0$で振子が静止している場合について考えると、

\[
    \theta_A = \theta_0 \cos(\omega_2 t), \theta_B = - \theta_0 \cos(\omega_2 t)
\]

が成り立つ。従って、同様にして$\omega_2$を求めることができる。

\subsubsection*{うなり}

$t=0$において、$\theta_A = \theta_0$, $\theta_b = 0$で振子が静止している場合について考える。

初期条件より$A = B = \theta_0/2$, $\phi_1 = \phi_2 = 0$となるので、

\begin{eqnarray*}
    \theta_A & = & \theta_0\cos\left(\frac{\omega_1 + \omega_2}{2}t\right)\cos\left(\frac{\omega_2 - \omega_1}{2}t\right) \\
    \theta_B & = & \theta_0\sin\left(\frac{\omega_1 + \omega_2}{2}t\right)\sin\left(\frac{\omega_2 - \omega_1}{2}t\right)
\end{eqnarray*}

である。これは、振幅振動を表している。この振幅振動の角周波数は$\omega_2 - \omega_1$であるから、その周期は$T = \frac{2\pi}{\omega_2 - \omega_1}$である。

\section{装置}

実験で用いた剛体振子実験装置について説明する。
この装置は、軽量のSUSパイプと0.35kgの重りからなる振り子、振子を吊るすための回転軸(SUSの細い棒)、そして軸を置く土台からなる。
振り子についているネジを締めることで、振子と軸を固定することができる。
さらに、土台には目盛りが付いており、振子が中心からどれだけ動いたかを測定することができる。また、軸受けから目盛りまでの距離は$R = 63.5[{\rm cm}]$である。

\section{方法}

まず、実験Aにより単振子の特性を調べた。
次に、実験Bにより回転軸の結合係数を調べた。
最後に、実験Cにより連成振動の特性を調べた。

\subsection*{実験A. 単振子の特性}

振子を回転軸に固定し、軸を土台に置いて自由に振動できるようにした。
そして、初期振幅を$x = 5[{\rm cm}]$とし、重りの中心と回転軸との間の距離$y$を変えながら、振子の振動数$f$を計測した。
振動数$f$を計測するときは、振子が静止状態から30往復する時間をストップウォッチで計測し、1往復にかかる時間$T$を求め、その逆数から振動数$f$を求めた。
また、$y$を変化させるときは、0.3[m]から0.6[m]まで5[cm]刻みで変化させた。

\subsection*{実験B. 回転軸の結合係数}

腕の長さ$y = 0.45[m]$の2つの振子A、Bを0.5mほどの間隔をあけて回転軸に固定した。この時、ねじれがないように注意した。
そして、外力が無い状態でポインタが示す目盛り$s_A(0)$、$s_B(0)$を記録した。

その後、振子Aのおもりを手で持って静かに座標$s_A$を変化させ、傾角$\theta_A = (s_A - s_A(0))/R$を変化させる。
それぞれの$s_A$ごとに$s_B$を読み取り、$\theta_B = (s_B - s_B(0)/R$を測定した。

\subsection*{実験C. 連成振動の特性}

実験Bと同様に実験器具を設定し、以下のa,b,cの3つの初期条件について、振子を振動させた。
a, bに関しては、振子が静止状態から30往復するまでの時間をストップウォッチにて計測した。cに関しては、振子を運動させ、片方の振子の振幅が0になってからもう一度振幅が0になるまでの時間をストップウォッチにて計測した。いずれの計測も、同じ条件で何度か実施した。

\subsubsection*{a. 角振動数$\omega_1$を与える初期条件}
$t=0$にて$s_A = 5[{\rm cm}]$,$s_B = 5[{\rm cm}]$の位置でポインタを置き、静止状態から静かに手を離す。

\subsubsection*{b. 角振動数$\omega_2$を与える初期条件}
$t=0$にて$s_A = 5[{\rm cm}]$,$s_B = - 5[{\rm cm}]$の位置でポインタを置き、静止状態から静かに手を離す。

\subsubsection*{c. うなりを与える初期条件}
$t=0$にて$s_A = 5[{\rm cm}]$,$s_B = 0[{\rm cm}]$の位置でポインタを置き、静止状態から静かに手を離す。

\section{データ}

\subsection*{実験A. 単振子の特性}

\tab{A}{csv/A.csv}{単振子の特性}{1}

\subsection*{実験B. 回転軸の結合係数}

振子に力を加えない状態のポインタの値は

\begin{eqnarray*}
    s_A(0) & = & 3.5[{\rm mm}] \\
    s_B(0) & = & 1.0[{\rm mm}]
\end{eqnarray*}

であった。

また、\tr{B}の$s_A$,$s_B$に関しては、測定値からそれぞれ$s_A(0)$, $s_b(0)$を引いた値である。

\tab{B}{csv/B.csv}{回転軸の結合係数}{3}

\subsection*{実験C. 連成振動の特性}

\subsubsection*{a. 角振動数$\omega_1$を与える初期条件}
\begin{itemize}
    \item[1回目] 39.17[s]
    \item[2回目] 39.19[s]
\end{itemize}

\subsubsection*{b. 角振動数$\omega_2$を与える初期条件}
\begin{itemize}
    \item[1回目] 35.56[s]
    \item[2回目] 35.54[s]
\end{itemize}

\subsubsection*{c. うなりを与える初期条件}
\begin{itemize}
    \item[1回目] 11.90[s]
    \item[2回目] 12.05[s]
    \item[3回目] 12.27[s]
    \item[4回目] 11.83[s]
    \item[5回目] 12.21[s]
\end{itemize}

\section{解析}
\section{考察}
\section{結論}

\end{document}
